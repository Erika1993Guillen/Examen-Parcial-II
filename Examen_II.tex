\documentclass[12pt,letterpaper]{article}\usepackage[]{graphicx}\usepackage[]{color}
%% maxwidth is the original width if it is less than linewidth
%% otherwise use linewidth (to make sure the graphics do not exceed the margin)
\makeatletter
\def\maxwidth{ %
  \ifdim\Gin@nat@width>\linewidth
    \linewidth
  \else
    \Gin@nat@width
  \fi
}
\makeatother

\definecolor{fgcolor}{rgb}{0.345, 0.345, 0.345}
\newcommand{\hlnum}[1]{\textcolor[rgb]{0.686,0.059,0.569}{#1}}%
\newcommand{\hlstr}[1]{\textcolor[rgb]{0.192,0.494,0.8}{#1}}%
\newcommand{\hlcom}[1]{\textcolor[rgb]{0.678,0.584,0.686}{\textit{#1}}}%
\newcommand{\hlopt}[1]{\textcolor[rgb]{0,0,0}{#1}}%
\newcommand{\hlstd}[1]{\textcolor[rgb]{0.345,0.345,0.345}{#1}}%
\newcommand{\hlkwa}[1]{\textcolor[rgb]{0.161,0.373,0.58}{\textbf{#1}}}%
\newcommand{\hlkwb}[1]{\textcolor[rgb]{0.69,0.353,0.396}{#1}}%
\newcommand{\hlkwc}[1]{\textcolor[rgb]{0.333,0.667,0.333}{#1}}%
\newcommand{\hlkwd}[1]{\textcolor[rgb]{0.737,0.353,0.396}{\textbf{#1}}}%

\usepackage{framed}
\makeatletter
\newenvironment{kframe}{%
 \def\at@end@of@kframe{}%
 \ifinner\ifhmode%
  \def\at@end@of@kframe{\end{minipage}}%
  \begin{minipage}{\columnwidth}%
 \fi\fi%
 \def\FrameCommand##1{\hskip\@totalleftmargin \hskip-\fboxsep
 \colorbox{shadecolor}{##1}\hskip-\fboxsep
     % There is no \\@totalrightmargin, so:
     \hskip-\linewidth \hskip-\@totalleftmargin \hskip\columnwidth}%
 \MakeFramed {\advance\hsize-\width
   \@totalleftmargin\z@ \linewidth\hsize
   \@setminipage}}%
 {\par\unskip\endMakeFramed%
 \at@end@of@kframe}
\makeatother

\definecolor{shadecolor}{rgb}{.97, .97, .97}
\definecolor{messagecolor}{rgb}{0, 0, 0}
\definecolor{warningcolor}{rgb}{1, 0, 1}
\definecolor{errorcolor}{rgb}{1, 0, 0}
\newenvironment{knitrout}{}{} % an empty environment to be redefined in TeX

\usepackage{alltt}
 \usepackage[left=2cm,right=2cm,top=2cm,bottom=2cm]{geometry}
\usepackage[ansinew]{inputenc}
\usepackage[spanish]{babel}
\usepackage{amsmath}
\usepackage{amsfonts}
\usepackage{amssymb}
\usepackage{dsfont}
\usepackage{multicol} 
\usepackage{subfigure}
\usepackage{graphicx}
\usepackage{float} 
\usepackage{verbatim} 
\usepackage[left=2cm,right=2cm,top=2cm,bottom=2cm]{geometry}
\usepackage{fancyhdr}
\pagestyle{fancy} 
\fancyhead[LO]{\leftmark}
\usepackage{caption}
\newtheorem{definicion}{Defincion}
\IfFileExists{upquote.sty}{\usepackage{upquote}}{}
\begin{document}

\begin{titlepage}
\setlength{\unitlength}{1 cm} %Especificar unidad de trabajo

\begin{center}
\textbf{{\large UNIVERSIDAD DE EL SALVADOR}\\
{\large FACULTAD MULTIDISCIPLINARIA DE OCCIDENTE}\\
{\large DEPARTAMENTO DE MATEMATICA}}\\[0.50 cm]

\begin{picture}(18,4)
 \put(7,0){\includegraphics[width=4cm]{minerva.jpg}}
\end{picture}
\\[0.25 cm]

\textbf{{\large Licenciatura en Estadistica}\\[1.25cm]
{\large Control Estadistico del Paquete R }\\[2 cm]
%\setlength{\unitlength}{1 cm}
{\large  \textbf{''EXAMEN PARCIAL II"}}\\[3 cm]
{\large Alumnas:}\\
{\large MARTHA YOANA MEDINA SANCHEZ}\\
{\large ERIKA BEATRIZ GUILLEN PINEDA}\\[2cm]
{\large Fecha de elaboracion}\\
Santa Ana - \today }
\end{center}
\end{titlepage}

\newtheorem{teorema}{Teorema}
\newtheorem{prop}{Proposicion}[section]

\lhead{Práctica 05}

\lfoot{LICENCIATURA EN ESTADISTICA}
\cfoot{UESOCC}
\rfoot{\thepage}
%\pagestyle{fancy} 

\setcounter{page}{1}
\newpage
Ejercicio #1
Hacer una funcion que dado un numero muestre el triangulo de pascal hasta ese numero
\begin{knitrout}
\definecolor{shadecolor}{rgb}{0.969, 0.969, 0.969}\color{fgcolor}\begin{kframe}
\begin{alltt}
\hlstd{TRIANGULOPASCAL} \hlkwb{<-} \hlkwa{function}\hlstd{(}\hlkwc{n}\hlstd{) \{}
    \hlkwa{for}\hlstd{(i} \hlkwa{in} \hlnum{0}\hlopt{:}\hlstd{(n}\hlopt{-}\hlnum{1}\hlstd{)) \{}
    \hlstd{s} \hlkwb{<-} \hlstr{""}
    \hlkwa{for}\hlstd{(k} \hlkwa{in} \hlnum{0}\hlopt{:}\hlstd{(n}\hlopt{-}\hlstd{i)) s} \hlkwb{<-} \hlkwd{paste}\hlstd{(s,} \hlstr{"  "}\hlstd{,} \hlkwc{sep}\hlstd{=}\hlstr{""}\hlstd{)}
    \hlcom{# La funcion paste() une todos los vectores de caracteres que se le suministran y construyen una sola cadena de caracteres.}
    \hlkwa{for}\hlstd{(j} \hlkwa{in} \hlnum{0}\hlopt{:}\hlstd{i) \{}
      \hlstd{s} \hlkwb{<-} \hlkwd{paste}\hlstd{(s,} \hlkwd{sprintf}\hlstd{(}\hlstr{"%3d "}\hlstd{,} \hlkwd{choose}\hlstd{(i, j)),} \hlkwc{sep}\hlstd{=}\hlstr{""}\hlstd{)}
    \hlstd{\}}
    \hlkwd{print}\hlstd{(s)}
  \hlstd{\}}
\hlstd{\}}
\hlkwd{TRIANGULOPASCAL}\hlstd{(}\hlnum{6}\hlstd{)}
\end{alltt}
\begin{verbatim}
## [1] "                1 "
## [1] "              1   1 "
## [1] "            1   2   1 "
## [1] "          1   3   3   1 "
## [1] "        1   4   6   4   1 "
## [1] "      1   5  10  10   5   1 "
\end{verbatim}
\end{kframe}
\end{knitrout}
Ejercicio #2
\begin{knitrout}
\definecolor{shadecolor}{rgb}{0.969, 0.969, 0.969}\color{fgcolor}\begin{kframe}
\begin{alltt}
\hlstd{med}\hlkwb{<-}\hlkwd{c}\hlstd{(}\hlnum{2}\hlstd{,}\hlnum{3}\hlstd{,}\hlnum{4}\hlstd{,}\hlnum{5}\hlstd{,}\hlnum{6}\hlstd{,}\hlnum{7}\hlstd{,}\hlnum{5}\hlstd{)}
\hlkwd{sort}\hlstd{(med)}
\end{alltt}
\begin{verbatim}
## [1] 2 3 4 5 5 6 7
\end{verbatim}
\begin{alltt}
\hlkwd{median}\hlstd{(med)}
\end{alltt}
\begin{verbatim}
## [1] 5
\end{verbatim}
\end{kframe}
\end{knitrout}
Ejercicio #3
\begin{knitrout}
\definecolor{shadecolor}{rgb}{0.969, 0.969, 0.969}\color{fgcolor}\begin{kframe}
\begin{alltt}
\hlstd{x}\hlkwb{<-}\hlkwd{c}\hlstd{(}\hlnum{1}\hlstd{,}\hlnum{2}\hlstd{,}\hlnum{3}\hlstd{,}\hlnum{4}\hlstd{,}\hlnum{5}\hlstd{)}
\hlstd{y}\hlkwb{<-}\hlkwd{c}\hlstd{(}\hlnum{0}\hlstd{,}\hlnum{2}\hlstd{,}\hlnum{5}\hlstd{,}\hlnum{6}\hlstd{)}
\hlstd{x}
\end{alltt}
\begin{verbatim}
## [1] 1 2 3 4 5
\end{verbatim}
\end{kframe}
\end{knitrout}
Ejercicio #4
\begin{knitrout}
\definecolor{shadecolor}{rgb}{0.969, 0.969, 0.969}\color{fgcolor}\begin{kframe}
\begin{alltt}
\hlstd{x}\hlkwb{<-}\hlkwd{c}\hlstd{(}\hlnum{1}\hlstd{,}\hlnum{2}\hlstd{,}\hlnum{3}\hlstd{,}\hlnum{4}\hlstd{,}\hlnum{5}\hlstd{,}\hlnum{6}\hlstd{,}\hlnum{7}\hlstd{,}\hlnum{8}\hlstd{)}
\hlkwd{sort}\hlstd{(x)}
\end{alltt}
\begin{verbatim}
## [1] 1 2 3 4 5 6 7 8
\end{verbatim}
\begin{alltt}
\hlkwd{length}\hlstd{(x)}
\end{alltt}
\begin{verbatim}
## [1] 8
\end{verbatim}
\begin{alltt}
\hlkwd{quantile}\hlstd{(x)}
\end{alltt}
\begin{verbatim}
##   0%  25%  50%  75% 100% 
## 1.00 2.75 4.50 6.25 8.00
\end{verbatim}
\end{kframe}
\end{knitrout}




\end{document}
